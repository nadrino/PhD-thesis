%% Copyright (C) 2018 Adrien Blanchet
%%

%%%%%%%%%%%%%%%%%%%%%%%%%%%%%%%%%%%%%%%%
%         Liste des packages           %
%%%%%%%%%%%%%%%%%%%%%%%%%%%%%%%%%%%%%%%%
\usepackage{blindtext}

%%%%%%%%%%%%%%%%%%%%%%%%%%%%%%%%%%%%%%%%
% Réglage des fontes et typo
\usepackage[utf8]{inputenc}
\usepackage[T1]{fontenc}

%\usepackage[square,sort&compress,sectionbib]{natbib}		% Doit être chargé avant babel
%\usepackage[sectionbib]{chapterbib}
%	\renewcommand{\bibsection}{\section{Références}}		% Met les références biblio dans un \section (au lieu de \section*)

\bibliographystyle{abbrv} % or try or unsrtnat
\usepackage{csquotes}
\usepackage[
    %backend=biber,
    style=numeric,
    citestyle=numeric,
    sortlocale=fr_FR,
    natbib=true,
    url=false, 
    doi=true,
    hyperref=true,
    backref=true,
    eprint=false,
    sorting=none
]{biblatex}
\addbibresource{bibliography/references.bib}
%\usepackage{chapterbib}
	%\renewcommand{\bibsection}{\section{Références}} % Met les références biblio dans un \section (au lieu de \section*)

%\usepackage[frenchb]{babel}
\usepackage[french]{babel}
\usepackage{lmodern}
\usepackage{ae,aecompl}										% Utilisation des fontes vectorielles modernes
\usepackage[upright]{fourier}

\usepackage{amsmath}
\usepackage[makeroom]{cancel} % pour les expréssions barrés

\usepackage{afterpage}
\usepackage{microtype} % Pour forcer le passage à la ligne des mots "insécables" sont en bout de phrase (typiquement en textit...)


%%%%%%%%%%%%%%%%%%%%%%%%%%%%%%%%%%%%%%%%
% Allure générale
\usepackage{enumerate}
\usepackage{enumitem}
\usepackage[section]{placeins}	% Place un FloatBarrier é chaque nouvelle section
\usepackage{epigraph}
\usepackage[font={small}]{caption}
\usepackage[francais,nohints]{minitoc}		% Mini table des matiéres, en francais
    \setcounter{secnumdepth}{3}
	\setcounter{minitocdepth}{3}			    % Mini-toc détaillées (sections/sous-sections)
\usepackage[notbib]{tocbibind}		        % Ajoute les Tables des Matiéres/Figures/Tableaux é la table des matiéres

%%%%%%%%%%%%%%%%%%%%%%%%%%%%%%%%%%%%%%%%
% Maths
\usepackage{amsmath}			% Permet de taper des formules mathématiques
\usepackage{amssymb}			% Permet d'utiliser des symboles mathématiques
\usepackage{amsfonts}			% Permet d'utiliser des polices mathématiques
\usepackage{nicefrac}			% Fractions 'inline'


%%%%%%%%%%%%%%%%%%%%%%%%%%%%%%%%%%%%%%%%
% Tableaux
\usepackage{multirow}
\usepackage{booktabs}
\usepackage{colortbl}
\usepackage{tabularx}
\usepackage{multirow}
\usepackage{threeparttable}
\usepackage{etoolbox}
	\appto\TPTnoteSettings{\footnotesize}
	\addto\captionsfrench{\def\tablename{{\textsc{Tableau}}}}


%%%%%%%%%%%%%%%%%%%%%%%%%%%%%%%%%%%%%%%%
% Eléments graphiques
\usepackage{graphicx}			% Permet l'inclusion d'images
\usepackage{subcaption}
\usepackage{pdfpages}
\usepackage{rotating}
\usepackage{pgfplots}
	\usepgfplotslibrary{groupplots}
\usepackage{tikz}
	\usetikzlibrary{backgrounds,automata}
	\pgfplotsset{width=7cm,compat=1.3}
	\tikzset{every picture/.style={execute at begin picture={
   		\shorthandoff{:;!?};}
	}}
	\pgfplotsset{every linear axis/.append style={
		/pgf/number format/.cd,
		use comma,
		1000 sep={\,},
	}}
\usepackage[framemethod=TikZ]{mdframed} % Pour le cadre de la page de garde
\usepackage{eso-pic}
\usepackage{import}


%%%%%%%%%%%%%%%%%%%%%%%%%%%%%%%%%%%%%%%%
% Mise en forme du texte
\usepackage{xspace}
\usepackage[load-configurations = abbreviations]{siunitx}
	\DeclareSIUnit{\MPa}{\mega\pascal}
	\DeclareSIUnit{\micron}{\micro\meter}
	\DeclareSIUnit{\tr}{tr}
	\DeclareSIPostPower\totheM{m}
	\sisetup{
	locale = FR,
	  inter-unit-separator=$\cdot$,
	  range-phrase=~\`{a}~,     	% Utilise le tiret court pour dire "de... à"
	  range-units=single,  		% Cache l'unité sur la premiére borne
	  }
	\sisetup{detect-all = true} % détecte si le texte autour est en italique par exemple
%\usepackage[squaren,Gray]{SIunits} % vieux package pour \unit -> incompatible avec siunitx

\usepackage[version=3]{mhchem}	% Equations chimiques
\usepackage{textcomp}
\usepackage{array}
\usepackage{hyphenat}


%%%%%%%%%%%%%%%%%%%%%%%%%%%%%%%%%%%%%%%%
% Navigation dans le document
\usepackage[
            %pdftex,
            pdfborder={0 0 0},
			colorlinks=true,
			linkcolor=blue,
			citecolor=red,
			pagebackref=false,
			]{hyperref}	% Créera automatiquement les liens internes au PDF
					% Doit étre chargé en dernier (Sauf exceptions ci-dessous)

%%%%%%%%%%%%%%%%%%%%%%%%%%%%%%%%%%%%%%%%
% Packages qui doivent étre chargés APRES hyperref
\usepackage[top=2.5cm, bottom=2cm, left=3cm, right=2.5cm,
			headheight=15pt]{geometry}

\usepackage{fancyhdr}			% Entéte et pieds de page. Doit étre placé APRES geometry
	\pagestyle{fancy}		% Indique que le style de la page sera justement fancy
	%\lfoot[\thepage]{} 		% gauche du pied de page
	\cfoot{\thepage} 			% milieu du pied de page
	%\rfoot[]{\thepage} 		% droite du pied de page
	\fancyhead[RO, LE] {}

\usepackage[acronym,xindy,toc,numberedsection,ucmark]{glossaries}
	\newglossary[nlg]{notation}{not}{ntn}{Notation} % Création d'un type de glossaire 'notation'
	\makeglossaries
	\loadglsentries{include/glossaire}			% Utilisation d'un fichier externe pour la définition des entrées (Glossaire.tex)
