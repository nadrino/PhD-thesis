%% Copyright (C) 2018 Adrien Blanchet
%%

%%%%%%%%%%%%%%%%%%%%%%%%%%%%%%%%%%%%%%%%
%            Conclusion                %
%%%%%%%%%%%%%%%%%%%%%%%%%%%%%%%%%%%%%%%%

\chapter*{Conclusion}
\label{chap:conclusion}
\addcontentsline{toc}{chapter}{\nameref{chap:conclusion}}
\markboth{\MakeUppercase{Conclusion}}{\MakeUppercase{Conclusion}}

\citationChap{
 J'ai voulu prouver que vouloir et comprendre suffisent, même à l'atome, pour triompher du plus formidable des despotes, l'infini.
 }{Victor Hugo}{Les Travailleurs de la mer}


\lettrine{A}{près} plus de 70 ans de progrès expérimentaux, la physique des neutrinos reste à la fois singulière, fuyante, et toujours à conquérir. L'entêtement toujours croissant des expérimentateurs et des théoriciens montre l'ampleur des défis que les neutrinos n'ont de cesse de poser. Pour les physiciens des particules, les neutrinos représentent avant tout un espoir qui pourrait nous montrer la voie vers une nouvelle physique au-delà du Modèle Standard. L'anomalie des antineutrinos de réacteur en est une incarnation : aurait-on manqué un état de masse des neutrinos ? Après huit années d'acharnement, la question du neutrino stérile vacille et semble s'écarter. En effet, les contraintes cosmologiques et les résultats des expériences auprès de réacteurs commerciaux à propos de l'oscillation $\overline{\nu}_e \rightarrow \overline{\nu}_s$ ne semblent pas en faveur de cette hypothèse. Cependant, la communauté attend les résultats des expériences auprès des réacteurs de recherche pour trancher la question.\\

\textsc{Stereo} a apporté sa pierre à l'édifice dès mars 2018 en proposant un contour d'exclusion qui aujourd'hui couvre la majeure partie de la zone pointée par la RAA. Ayant misé sur la technologie de détection des liquides scintillateurs dopés au gadolinium, les conditions de stabilité et de rendement lumineux ont été atteintes. Malgré les deux \textit{buffers} défaillants ainsi que l'évolution des fuites de lumière entre cellules durant la première phase de prise de données, la méthode de reconstruction en énergie a permis d'assurer une référence identique pour chaque cellule dans les données et la simulation avec la source de calibration $\ce{^{54}Mn}$. Elle a également garanti la stabilité de la réponse énergétique à mieux que $1\%$ pendant l'ensemble des deux phases de prise de données. Cet aspect est essentiel pour l'application de la procédure de soustraction des bruits de fond corrélés. En effet, la stabilité de l'échelle en énergie garantie à la fois la validité de la soustraction de deux spectres mesurés pendant des périodes différentes (ON et OFF), mais aussi l'efficacité des coupures topologiques employées pour isoler au mieux le signal neutrino. Par ailleurs, les non-linéarités de la réponse en énergie ont été ajustées dans la simulation via le paramètre de Birks qui règle l'amplitude du \textit{quenching} du liquide scintillateur. Cet effet est le principal responsable des non-linéarités en dessous de $\sim \SI{2}{MeV}$ et son ajustement est important pour faire correspondre l'ancrage manganèse à $\SI{835}{keV}$ à la plage en énergie du spectre des positrons allant de 2 à $\SI{8}{MeV}$. Enfin, l'estimation des incertitudes liées à la reconstruction en énergie a été menée en simulant les gammas issus des captures de neutrons cosmogéniques sur des noyaux d'Hydrogène (n-H). Ces gammas monoénergétiques fournissent une raie à $\SI{2.2}{MeV}$ dont les dépôts d'énergie sont plus homogènes que les sources déployées dans les tubes de calibration. Afin d'obtenir une répartition des dépôts d'énergie fidèle, ces gammas ont été générés à partir de simulations de muons d'origine cosmique. Celles-ci ont montré qu'à part une légère surpopulation dans le fond du détecteur, les dépôts d'énergie sont similaires à ceux attendus par les neutrinos. La comparaison données-simulation des valeurs les plus probables en énergie reconstruite des distributions n-H a dévoilé un désaccord systématique à hauteur d'1\% sur la moyenne des six cellules. Ce décalage a été attribué aux biais résiduels de la description de la collection de lumière dans le MC. L'effet dominant a été identifié comme de faibles termes d'absorption au niveau des parois réfléchissantes et des tubes de calibration. Leur implémentation a déjà permis une réduction significative des effets résiduels entre données et MC au niveau des charges brutes. Cette mise à niveau de la simulation exige un retraitement complet l'échelle en énergie, en cours pendant l'écriture de ce manuscrit.\\

La seconde moitié de la thèse a été consacrée à l'inférence statistique et à la génération des contours d'exclusion. Puisque les spectres positron sont mesurés dans six cellules qui couvrent des distances de propagation différentes, les motifs d'oscillations vers un neutrino stérile avec $\Delta m^2 \sim \SI{1}{eV^2}$ sont déphasés. C'est précisément ce déphasage relatif entre cellules qui procure une sensibilité au neutrino stérile tout en s'affranchissant des prédictions extérieures. Pour la publication des premiers résultats en 2018, l'analyse statistique a été menée en effectuant le rapport des spectres mesurés dans les cellules 2, 3, 4, 5 et 6 sur celui de la cellule 1. Cependant afin de maximiser le pouvoir de discrimination de \textsc{Stereo}, une nouvelle méthode de comparaison relative a été développée. Celle-ci consiste à laisser des paramètres libres épouser une forme globale du spectre mesuré, c'est-à-dire avec des paramètres communs à toutes les cellules. L'inférence statistique est alors établie à partir des distorsions résiduelles. Une attention particulière a été portée sur la propagation des incertitudes statistiques et systématiques via des matrices de covariance. Les contours de sensibilité ont été tracés à l'aide des prescriptions de la méthode de Feldman et Cousins (1999) : comparaison avec un $\Delta \chi^2$ de chaque scénario de neutrino stérile avec le meilleur ajustement dans l'espace des paramètres $(\Delta m^2_{41}, \textrm{sin}^2(2\theta_{14}))$. La non-linéarité des motifs d'oscillations en $\Delta m^2$ a contraint les études à ne pas utiliser des lois normales de $\chi^2$ pour déduire les niveaux de confiance. Pour cela, des pseudo-expériences ont été générées à chaque couple de paramètres $(\Delta m^2_{41}, \textrm{sin}^2(2\theta_{14}))$ afin de construire les PDFs appropriées en $\Delta \chi^2$. Chaque hypothèse d'oscillation testée avec les vraies données s'est vue attribuée d'un niveau de confiance en comparant son $\Delta \chi^2$ avec la PDFs associée. Finalement cette étude a abouti à la génération de contours d'exclusion sur le plan $(\Delta m^2_{41}, \textrm{sin}^2(2\theta_{14}))$. Un effort particulier a été porté aux vérifications des résultats de calculs ainsi qu'au développement et optimisation de l'algorithme de minimisation sur-mesure pour générer les PDF en $\Delta \chi^2$, non substituable avec des minimiseurs classiques par gradient à cause de la non-linéarité du problème en $\Delta m^2$.\\

Les derniers contours d'exclusion ont été présentés à la conférence de Moriond 2019 \cite{docdb849} et ont montré que \textsc{Stereo} a été capable de rejeter le scénario le plus probable (\textit{best-fit}) de Mention \textit{et al.} à 99,8\% de niveau de confiance, avec plus de 65 000 neutrinos mesurés et un rapport signal sur bruit moyen de $0,9$. À l'heure actuelle, le détecteur continue de récolter des neutrinos près du réacteur de l'ILL pour améliorer la précision statistique, et de nouvelles études sont entreprises en vue de confronter les prédictions extérieures avec les mesures. Les dernières estimations des effets systématiques montrent que \textsc{Stereo} a la capacité de mesurer le flux total de neutrinos avec une des meilleures précisions du marché. D'autre part, les analyses en forme du spectre d'antineutrinos issus de la décroissante de l'$\ce{^{235}U}$ ont révélé le besoin d'approfondir la propagation des contraintes sur l'échelle en énergie pour délibérer sur la pertinence des prédictions d'Huber. L'ajout du spectre de décroissance bêta du $\ce{^{12}B}$ dans les données de calibration est une avancée cruciale dans ce sens. Le formalisme développé pour contraindre les distorsions possibles de l'échelle en énergie semble capable d'atteindre la précision requise sur toute la gamme du spectre neutrino. Parallèlement, ce formalisme aura le potentiel de fournir une nouvelle référence de spectre antineutrino associé à l'$\ce{^{235}U}$ et participera à la compréhension des anomalies.\\
