%% Copyright (C) 2018 Adrien Blanchet
%%

%%%%%%%%%%%%%%%%%%%%%%%%%%%%%%%%%%%%%%%%
%            Remerciements             %
%%%%%%%%%%%%%%%%%%%%%%%%%%%%%%%%%%%%%%%%

\chapter*{Remerciements}
\label{chap:remerciements}

\lettrine{I}{l y a 8 ans}, c'était presque par hasard que je m'étais retrouvé en première année de licence de physique à l'université de Tours, avec en tête un tout autre projet d'études. Il faut dire que les multiples rencontres avec des chercheurs savent parfaitement donner l'envie à tout étudiant réceptif de poursuivre dans cette voie. Plutôt que de s'engager dans des études d'ingénieurs, j'ai choisi d'intégrer le magistère d'Orsay en vue de s'orienter vers la recherche en physique fondamentale (est-ce peut-être une erreur de jugement ? {\tt ;)}). Bien que cette formation fut très riche, intense et parfaitement juste, ma passion pour la physique expérimentale n'aurait pu être rassasiée sans achever cette aventure par trois années supplémentaires en préparant une thèse de doctorat. Au travers ces quelques lignes qui tenterons de retranscrire pèle-mêle mes sentiments à l'issue de la soutenance, je compte exprimer ma gratitude à tous ces individus, ces caractères sans qui la route ne m'aurait pas procuré tant de bonheur.\\

Commençons cette parade dithyrambique par la fin, c'est-à-dire par les membres de mon juri de thèse qui ont absolument tous, et je tiens à les remercier très chaleureusement pour cela, lu ce pavé dans son intégralité. J'aimerais tout d'abord adresser mes sincères remerciements à Madame la présidente du juri, Marie-Hélène \textsc{Schune}, pour avoir accepté ma proposition ainsi que de m'avoir fourni quelques corrections au manuscrit avec grande générosité. À François \textsc{Mauger}, rapporteur de ma thèse, je voulais saluer l'immense temps consacré à scruter les moindres détails de fond et de forme du manuscrit, mais également pour partager des conseils indispensables pour la suite. De la même manière, je souhaitais remercier Antonin \textsc{Vacheret}, rapporteur de ma thèse, pour m'avoir apporté ses impressions et son recul sur la question du neutrino stérile. J'aimerais enfin saluer Thomas \textsc{Mueller} ainsi que Karsten \textsc{Heeger} pour le soutien qu'ils m'ont accordé pendant comme après la soutenance.\\

Cette thèse n'aurait été rendu possible sans l'appui de l'Irfu, sous la direction d'Anne-Isabelle \textsc{Etienvre}, c'est pourquoi je veux manifester ma reconnaissance à l'égard des chefs du DPhN : Héloïse \textsc{Goutte}, Jacques \textsc{Ball} et plus particulièrement Franck \textsc{Sabatier} avec qui les discutions ponctuelles échangées tout au long de ces trois années m'ont apportées des conseils essentiels ainsi que l'ouverture d'esprit pour envisager mon orientation en postdoc dans la physique des neutrinos d'accélérateurs. J'en profite également pour remercier chaleureusement Danielle \textsc{Coret} pour sa bonne humeur à toute épreuve, ses conseils et ses rappels concernant les procédures administratives auxquelles chaque doctorant est confronté. Pareillement, j'adresse un grand merci à Isabelle \textsc{Richard} pour sa franchise, son efficacité et son assistance pour les missions nombreuses (et soudaines!) que j'ai effectuées à Grenoble.\\

Je voudrais maintenant remercier du fond du c\oe ur mon directeur de thèse, David \textsc{Lhuillier}, pour la pertinence de son encadrement qui m'a considérablement apporté tant sur le plan académique que humain. Ces trois années à travailler étroitement ensemble sur \textsc{Stereo} (entre autres) sont passées en un éclair, dans le contexte d'un cas scientifique extrêmement stimulant et très excitant. Je tenais absolument à souligner la passion avec laquelle David s'acharne quotidiennement pour mener à bien et surtout comprendre chaque aspect, chaque détail des projets dans lesquels il est impliqué. Mais je crois qu'avant tout, David est un grand enfant sans cesse émerveillé par la physique expérimentale, qu'il s'efforce d'enseigner à chaque étudiant en stage, en thèse, ou encore à la MJC de Gif par le biais de diverses petites expériences faites main. J'ai pu particulièrement apprécier sa disponibilité et son écoute, sublimée par une aisance à expliquer n'importe quel phénomène physique, avec qui plus est, une intuition se mettant à chaque fois au diapason de son interlocuteur. J'ai ainsi pu appréhender chaque jour de ma thèse avec délectation, et j'espère que beaucoup d'étudiants auront eux aussi la chance de bénéficier de cette exaltation, cet ardeur inexorable qu'il rayonne.\\

J'en viens naturellement au groupe de physiciens du LEARN que j'ai côtoyé quotidiennement chaque fois avec un plaisir immense. J'aimerais tout d'abord remercier très sincèrement Alain \textsc{Letourneau} pour les nombreuses discutions de physique que nous avons eu, que ce soit à propos de \textsc{Stereo} ou d'autres sujets plus transverses. Chacun de ces échanges m'ont apporté une hauteur de vue et m'ont particulièrement été bénéfiques pour comprendre certains enjeux de l'expérience. Je ne pourrais parler de la vie au laboratoire sans mentionner Thomas \textsc{Materna}, collaborateur \textsc{Stereo} par alternance. Toujours prêt à donner un coup de main pour se plonger dans les entrailles du code de simulation, j'ai pu apprécier sa lucidité et son humour pour répondre à des problématiques multiples et variées, allant d'aspects très techniques à des questions d'analyse mathématiques plus abstraites. Incontournable protagoniste de mes deux premières années en thèse, Aurélie \textsc{Bonhomme} a su parfaitement accompagner mon arrivée dans la collaboration. Avec son intelligence débordante d'humilité, Aurélie a su répondre à tous mes questionnements et m'a beaucoup apporté tant sur la physique que sur le plan personnel. J'ai toujours considéré Aurélie comme un modèle, une grande s\oe ur en quelque sorte, et je lui dois aujourd'hui beaucoup. J'ai soudain une pensée pour notre cher camarade Alessandro \textsc{Minotti}, complétant la force de travail saclaysienne (les 3A!) sur le \textsc{Stereo} des premières années, qui savait nous donner la force et le courage nécéssaire pour affronter chaque défi. Les discussions riches, profondes et aux sujets divers passées aux Lock Groove ou à la maison pour nous faire découvrir ses talents de pizzaïolo, ponctuaient chaque moment passés ensemble, et j'aurais souhaité que cela dure bien davantage. Désormais, la nouvelle génération d'étudiant a pris la relève avec tout autant d'entrain que la précédente. Merci à Vladimir \textsc{Savu}, qui s'est vu assigner la lourde tâche de poursuivre le développement du code de \textsc{Stereo} sentant encore \textit{rush} du \textit{commisionning}. Sa curiosité inébranlable pour la physique fondamentale, ainsi que sa passion pour la littérature ont rendu la transition très agréable. Je pense évidement à mon successeur à l'enthousiasme contagieux, Rudolph \textsc{Rogly}, qui a embrassé sa thèse avec entêtement, sans cesse à l'affut de comprendre chaque technique employée dans l'analyse de données.\\

Mais le LEARN ce n'est pas que \textsc{Stereo}, et par là je voudrais exprimer ma gratitude envers Loïc \textsc{Thulliez}, que j'ai connu à la fois étudiant et permanent du laboratoire, avec qui les échanges à propos d'expositions d'art contemporain apportaient un vent d'air frais à notre quotidien submergé par les couacs imposés par le centre de calcul de Lyon. Merci également à Diane \textsc{Dore}, notre chercheuse Québécoise préférée dont l'adorable inquiétude pour sa fille m'a rappelé ces moments de doutes qu'accompagne chaque adolescent à la sortie du Baccalauréat. J'adresse aussi un grand merci à Quentin \textsc{Deshayes}, chercheur Normand en postdoc à Saclay, qui m'a fait bénéficier généreusement de son expérience avec la culture Japonaise. Enfin, pas techniquement au LEARN mais régulièrement membre de notre groupe de midi, je voulais remercier la bienveillante Nicole \textsc{D'Hose}, dont la tendresse qu'elle propage s'étend jusqu'aux chats sauvages de l'Irfu.\\

Toujours au DPhN mais dans une tout autre dynamique, je voudrais faire un clin d'\oe il aux semaines passés à encadrer les travaux de laboratoire introduisant le master 2 NPAC aux futurs physiciens. J'adresse un immense merci à Stefano \textsc{Panebianco}, qui m'a accordé sa confiance pour guider et mettre à l'épreuve les étudiants des TPs muons et discrimination neutron-gamma. Et puisque l'irremplaçable devait être remplacé l'année suivante, la responsabilité des TLs est revenue aux nouveaux permanents du DPhN : Marine \textsc{Vanderbrouck} et Maxime \textsc{Defurne}. Je souhaitais leur adresser ma gratitude pour leur écoute et leur compréhension à mon égard, et les félicite au passage pour le stress qu'ils ont enduré en vue d'assurer une transition sans faille. Je dois bien avouer qu'ils sont tous les trois (avec Loïc) une des raisons qui me poussent à tout donner pour pouvoir consacrer ma carrière à la recherche en physique.\\

Avant de quitter le DPhN, j'ai une pensée pour les \og précaires \fg{}, source vive de doctorants au laboratoire. Lors de mes débuts en thèse, nous avions la chance de bénéficier d'un corp solidaire et une atmosphère détendue dont l'impulsion était donnée par Nancy \textsc{Hupin}. Il me semble que la relève ait été parfaitement assuré par Zoé \textsc{Favier}, toujours enthousiaste et pleine de vie. Je me souviendrais évidement de la présence iconique d'Antoine \textsc{Vidon}, pour sa bonne humeur omniprésente, mais aussi et surtout pour ces conversations allant de Pierre Bourdieu à des anecdotes plus frivoles. Egalement je remercie Nicolas \textsc{Pierre} pour ces discutions très stimulantes dans le bus du soir, Christopher \textsc{Filosa} l'ambassadeur du sud toujours prêt à jouer de sarcasmes dans le but de détendre l'atmosphère, Saba \textsc{Ansari} pour avoir partagé à la fois sa bonne humeur ses doutes, Pierre \textsc{Arthuis} pour sa répartie sans limites, Brian \textsc{Ventura} intrépide devant chaque défis de COMPASS et lui aussi élu désigné par l'université Paris-Saclay, Charles-Joseph \textsc{Naïm} pour sa curiosité ainsi que les échanges constructifs et profonds que nous avons entretenus, ou encore Benjamin \textsc{Manier} pour sa passion pour les divertissements et sports électroniques si chers à la Corée du Sud. J'adresse à l'occasion un clin d'\oe il à Julien \textsc{Ripoche} camarade de NPAC à mi-temps au DPhN, toujours partant pour une discussion autour d'un café  dans l'après-midi. \\

À l'Irfu, la physique des neutrinos s'étend aussi (et surtout) jusqu'au DPhP où j'ai eu l'occasion et la chance de profiter des précieux conseils de Guillaume \textsc{Mention}. Je lui adresse mes remerciement tout particulièrement pour son assistance sur l'analyse de \textsc{Stereo} et sa pédagogie qui m'a aidé à comprendre les fondements des analyses statistiques fréquentistes en physique des oscillations. Je salut au passage Mathieu \textsc{Vivier} pour son humanité matérialisée par les promptes discutions que nous avons entretenues régulièrement depuis les TLs lors de mon M2. Je voulais aussi remercier nos collègues du DEDIP, ayant été en charge de la construction du détecteur interne, avec qui nous nous rencontrions annuellement le temps d'un pot, organisé par Carline \textsc{Lahonde}, dans le but d'annoncer les résultats et avancements de l'expérience. J'en profite par la même occasion pour adresser ma reconnaissance à l'égard de Patrick \textsc{Champion} et Yves \textsc{Penichot} avec qui nous avons partagés quelques verres et quelques briques de plomb à Grenoble.\\

En passant par Grenoble, il me semble indispensable de mentionner nos collègues du Laboratoire de Physique Subatomique et de Cosmologie (LPSC), avec qui j'ai longuement collaboré lors des nombreux \textit{shifts} à l'ILL. Tout d'abord, je souhaitais remercier chaleureusement Anne \textsc{Stutz} pour toute l'aide et la disponibilité qu'elle m'a accordé. Je réservais également un très grand merci à Jean-Sebastien \textsc{Réal} ainsi que Jacob \textsc{Lamblin} pour tout leur soutient sur la reconstruction en énergie et l'analyse statistique. Enfin, il me vient une pensée à notre camarade lecteur du Monde Diplomatique, Serge \textsc{Kox} aujourd'hui en situation de travail \og libre \fg{} si cher à Bernard Friot.\\

Pour terminer, bien que j'ai choisi de poursuivre dans la voie des neutrinos pour le moment, je voudrais rendre hommage à quelques individus qui ont stimulés ma passion pour les ondes gravitationnelles. Merci à Thierry \textsc{Foglizzo} pour l'organisation des \textit{LISA clubs} qui offrent la possibilité de découvrir la physique des ondes gravitationnelles au chercheurs de l'Irfu. Egalement merci à Antoine \textsc{Petiteau} qui est venu nous rendre visite plusieurs fois à l'Irfu pour échanger des tutoriels sur l'analyse de données LISA et qui m'a aiguillé sur les pistes de post-doctorat possibles. Enfin pendant ma thèse j'ai eu la chance de pouvoir participer à l'école des Houches sur les onde gravitationnelles, et je souhaitais tout particulièrement remercier Marie-Anne \textsc{Bizouard} qui m'a permis de trouver le financement nécéssaire et qui m'a appris beaucoup sur la physique des ondes gravitationnelles toujours avec exaltation. J'espère de tout c\oe ur pouvoir à l'avenir m'investir davantage sur l'un de ces projets fantastiques, ces pari fous, en vue d'apporter une pierre à l'édifice de cette belle physique en plein essor.\\




% Aurélie /
% Alessandro /
% Vladimir Savu /
% Rudolph Rogly /

% Nicole D'Hose /
% Diane /
% Quentin /
% Loic /

% TLs
% Stefano /
% Marine /
% Maxime /

% DPhP
% Guillaume Mention /
% Mathieu Vivier /

% Service Technique
% Caroline LAHONDE-HAMDOUN /
% Patrick Champion /  Penichot Yves /

% Grenoble
% Serge /
% Anne /
% Jean-Sebastien /
% Jacob /



% Ondes gravitationnelles
% Thierry Foglizzo
% Antoine Petiteau
% Herve MOUTARDE
% Marie-Anne Bizouard
% Nelson Christensen

%\blindtext


