%% Copyright (C) 2018 Adrien Blanchet
%%

%%%%%%%%%%%%%%%%%%%%%%%%%%%%%%%%%%%%%%%%
%            Introduction              %
%%%%%%%%%%%%%%%%%%%%%%%%%%%%%%%%%%%%%%%%

\chapter*{Introduction}
\label{chap:introduction}
\addcontentsline{toc}{chapter}{\nameref{chap:introduction}}
\markboth{\MakeUppercase{Introduction générale}}{\MakeUppercase{Introduction générale}}

% Des masses, des directions, des espaces limités dans le grand espace, l’univers. Des masses différentes, légères, lourds, moyennes,—indiquées par des variations de grandeur ou de couleur—des directions—vecteurs représentant vitesses, vélocités, accélérations, forces, etc... Ces directions faisant entre elles des angles significatifs, et des sens, définissant ensemble une grande résultante ou plusieurs. Des espaces, des volumes, suggérés par les moindres moyens opposés à leur masse, ou même les contenant, juxtaposés, percés par des vecteurs, traversés par des vitesses.

% Que ce soit, non seulement un instant \og momentané \fg{}, mais une loi physique de variation entre les événements de la vie.

\citationChap{
Chaque élément pouvant bouger, remuer, osciller, aller et venir dans ses relations avec les autres éléments de son univers. [...] Pas d’extractions, des abstractions. Des abstractions qui ne ressemblent à rien de la vie, sauf par leur manière de réagir.
}{Alexander Calder}{Comment réaliser l’art ?}

\lettrine{L}{es} grands bouleversements de la physique fondamentale ont été imposés aux physiciens, souvent contre leur gré, par une variété de données expérimentales qui refusaient obstinément d'être prises en compte par les théories dominantes. Le neutrino en est le parfait exemple. À cause d'un spectre de désintégration beta opiniâtrement continu, il a du être postulé désespérément par Pauli en 1931 pour n'être finalement confirmé expérimentalement qu'au milieu des années 1950. Au cours de ces soixante dernières années, les immenses progrès instrumentaux ont permis d'en savoir davantage sur cette particule mystérieuse, mais ont aussi apporté leur lot de questions ouvertes : quels mécanismes sous-jacents confèrent une masse aux neutrinos ? Le neutrino est-il sa propre antiparticule ? Comment expliquer les anomalies persistantes ?\\

En 2015, le prix Nobel de physique a été attribué à Takaaki Kajita et Arthur B. McDonald pour la découverte des oscillations des neutrinos, qui montrent que deux au moins des trois neutrinos connus ont une masse non nulle. Aujourd'hui, un très grand nombre de résultats d'oscillations obtenus avec diverses configurations et techniques expérimentales peuvent être interprétés dans le cadre de trois neutrinos massifs actifs, dont les états propres de masse et de saveur sont reliés par une matrice de mélange unitaire $3\times 3$, la matrice Pontecorvo-Maki-Nakagawa-Sakata (PMNS), paramétrée par trois angles de mélange $\theta_{12}$, $\theta_{23}$, $\theta_{13}$ et une phase violatrice de la symétrie CP: $\delta_{CP}$. La probabilité d'oscillation est également régie par les différences de masse au carré $\Delta m^2_{ij} = m^2_i - m^2_j$, où $m_i$ est la masse du $i$ème état propre de masse du neutrino. Le dernier angle de mélange, $\theta_{13}$, vient d'être mesuré avec précision par les expériences auprès des réacteurs. Sa valeur proche de la limite supérieure préalablement établie ouvre la voie vers les mesures de la phase de violation CP dans le secteur leptonique.\\

%Au cours des prochaines années, un riche programme d'efforts expérimentaux continuera de porter ses fruits et de s'attaquer aux trois pièces manquantes du puzzle, à savoir la détermination de l'octant et la valeur précise de l'angle de mélange $\theta_{23}$, le dévoilement de l'ordre des masses de neutrinos ($m_1 < m_2 < m_3$ ou $m_3 < m_1 < m_2$) et la mesure de la phase de violation du CP $\delta_{CP}$.

En parallèle, des études sur les spectres d'antineutrinos des réacteurs ont souligné une déviation entre la prédiction et les mesures de flux réalisées jusqu'alors. Ce désaccord a été baptisé \og anomalie des antineutrinos de réacteurs \fg{}, aussi désigné sous le nom de RAA pour anomalie des antineutrinos de réacteurs (\textit{Reactor Antineutrino Anomaly}). L'introduction d'un nouveau neutrino dit \og stérile \fg{}, c'est-à-dire qui ne se couple pas avec l'interaction faible,  offrirait une solution à cette anomalie. Si cette hypothèse s'avère juste, alors il s'agira d'un véritable coup de tonnerre qui retentirait depuis la physique des particules jusqu'en cosmologie. En l'absence de signal de neutrino stérile, il faudra comprendre les biais sur la prédiction des spectres qui ont donnés lieu à la RAA.\\

C'est dans ce contexte que se place l'expérience \textsc{Stereo} qui vise à tester l'hypothèse du neutrino stérile auprès du réacteur de l'ILL de Grenoble. Puisqu'il ne se couple à aucune interaction hormis la gravitation, il ne peut être détecté directement. En revanche sa présence se manifeste lors de la propagation des 3 états actifs de neutrinos qui peuvent se mélanger vers un état supplémentaire. Le signal recherché avec \textsc{Stereo} est alors une probabilité de disparition des antineutrinos électroniques issus du réacteur sous la forme d'un nouveau motif d'oscillation de période spatiale de quelques mètres. Le principe de \textsc{Stereo} repose sur six cellules de détection identiques, remplies de liquide scintillateur dopé au Gadolinium, disposées entre 9 et 11,5 m de distance du coeur du réacteur de recherche de l'ILL. Ce dispositif a été retenu dans le but de fournir une comparaison relative des spectres mesurés dans chaque cellule, permettant de délibérer sur la question du neutrino stérile tout en s'affranchissant de prédictions extérieures. À la manière des expériences pionnières de neutrinos de réacteurs, le principe de détection est basé sur la double identification d'un positron et d'un neutron dans un intervalle de temps de quelques dizaines de microsecondes. Le détecteur a commencé la prise de données en novembre 2016, et les premiers résultats ont été publiés dès 2018.\\

Ce manuscrit est composé de sept chapitres qui retracent les principaux aspects de l'expérience, centrés sur les travaux réalisés pendant cette thèse. Dans une première partie, le contexte de \textsc{Stereo} est introduit en établissant un bref portrait du neutrino par l'histoire de sa découverte et de ses propriétés, avec un accent particulier sur le mécanisme d'oscillation. Les récentes anomalies sont présentées ensuite, ainsi qu'un panorama des résultats déjà publiés à ce jour par les différents projets. L'expérience \textsc{Stereo} est détaillée ensuite dans le chapitre 2, où les conditions expérimentales sont exposées suivi d'une synthèse de la technologie employée pour détecter les neutrinos.\\

Les chapitres suivants sont focalisés  sur l'analyse de données. D'abord, le chapitre 3 est dédié à la description du détecteur dans \textsc{Geant4}, et des simulations des bruits de fond cosmogéniques ainsi que des neutrinos. Ces données simulées sont cruciales pour établir la liste des erreurs systématiques ou encore pour construire les spectres neutrinos auxquels sont confrontées les véritables données. D'autre part, en vue de faire face aux évolutions temporelles de la réponse en énergie du détecteur, un algorithme de reconstruction énergétique dédié a été développé pour établir une échelle en énergie commune entre les données et la simulation. Cet aspect de l'analyse est détaillé en profondeur dans le chapitre 4 suivi d'une discussion sur l'estimation des incertitudes systématiques. Le chapitre 5 est consacré à l'extraction du signal neutrino, où l'algorithme de recherche de paires corrélées en temps est reporté après avoir exposé les coupures de sélection qui servent à purifier l'échantillon de candidats neutrinos. Les deux méthodes de reconstruction des spectres positron sont présentées enfin. L'inférence statistique en physique des oscillations nécessite certains éclaircissements afin de justifier les pratiques employées pour générer les contours de sensibilité dans \textsc{Stereo}. Le chapitre 6 s'attelle à cette tâche en commençant par une approche générale (mais pragmatique) sur ce qu'est une déduction statistique. La discussion se poursuit par le formalisme qui a été choisi pour tester l'hypothèse du neutrino stérile uniquement à l'aide des distorsions relatives des spectres entre cellules. La propagation des erreurs est détaillée ensuite pour finir sur quelques exemples de contours de sensibilité. Le dernier chapitre présente les résultats de \textsc{Stereo} en phase 2 dans l'état actuel de l'analyse. La mise à l'épreuve des spectres mesurés a permis d'exclure une bonne partie des paramètres d'oscillation vers un neutrino stérile autorisés par la RAA. Les perspectives pour des mesures absolues des spectres, c'est-à-dire avec des prédictions extérieures, sont enfin discutées.\\

% Les travaux qui ont été menés pendant ces trois ans de thèse sont majoritairement représentés les chapitres 3, 4 et 6. Les deux premières années ont été essentiellement consacrées à l'ajustement des paramètres de la simulation et au développement des outils d'analyse pour reconstruire l'énergie déposée par les particules dans le liquide. La compréhension du détecteur et des processus physiques qui engendrent les divers bruits de fond a été capitale pour achever [...]

\bigbreak

